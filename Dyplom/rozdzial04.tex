\chapter{Podsumowanie pracy}

Głównym celem pracy było zaprojektowanie i wdrożenie aplikacji internetowej typu eBOK dla wspólnoty mieszkaniowej. Cel ten udało się osiągnąć. Zaprojektowany i zaimplementowany system „Harmony Home Net” jest narzędziem, które może wspierać proces zarządzania wspólnotą, wspomagać komunikację pomiędzy mieszkańcami a administracją, automatyzować kluczowe zadania związane z obsługą nieruchomości. \\[-10pt]

\noindent\textbf{Wyniki pracy}\newline
\noindent W pracy przedstawiono pełny cykl życia oprogramowania, począwszy od analizy wymagań funkcjonalnych i niefunkcjonalnych, poprzez projektowanie architektury, aż po implementację i testy prototypu. System został zaprojektowany w oparciu o nowoczesne technologie, takie jak:
\begin{itemize}
    \item \textbf{Backend:} Spring Boot -- zapewniający elastyczność, bezpieczeństwo i wydajność aplikacji,
    \item \textbf{Frontend:} Next.js -- umożliwiający responsywność interfejsu i optymalizację SEO,
    \item \textbf{Baza danych:} PostgreSQL -- gwarantująca integralność i spójność danych,
    \item \textbf{Konteneryzacja:} Docker -- wspierający skalowalność i łatwość wdrażania systemu.
\end{itemize}
System oferuje funkcje dostosowane do potrzeb mieszkańców, administracji i administratorów:
\begin{itemize}
    \item \textbf{Zarządzanie użytkownikami} -- Kompleksowy system ról i poziomów dostępu,
    \item \textbf{Obsługa płatności} -- Możliwość przeglądania należności, realizacji płatności online oraz harmonogramowania przypomnień,
    \item \textbf{Zgłoszenia techniczne} -- Funkcja zgłaszania i monitorowania usterek,
    \item \textbf{Zarządzanie dokumentami} -- Dostęp do umów, regulaminów i faktur,
    \item \textbf{Głosowania} -- Mechanizmy organizacji głosowań i walidacji wyników,
    \item \textbf{Forum mieszkańców} -- Uproszczona platforma komunikacji wspólnoty.
\end{itemize}

\noindent\textbf{Wnioski}\newline
Stworzony system stanowi solidną bazę do rozwoju nowoczesnych aplikacji wspierających obsługę wspólnot mieszkaniowych. Realizacja tego projektu umożliwiła zdobycie praktycznych umiejętności w zakresie projektowania systemów informatycznych, integracji zewnętrznych usług oraz testowania oprogramowania. Ponadto, prototyp spełnia swoje założenia projektowe i jest gotowy do dalszego rozwijania w kierunku pełnoprawnej aplikacji produkcyjnej, która może znacząco poprawić efektywność zarządzania wspólnotami mieszkaniowymi. Rozwój ten może nastąpić w kilku kluczowych obszarach:
\begin{itemize}
    \item pełna integracja z zewnętrznymi systemami, np. bankowości czy urządzeń pomiarowych,
    \item rozbudowa funkcji analitycznych, takich jak generowanie zaawansowanych raportów,
    \item wdrożenie bardziej zaawansowanego interfejsu użytkownika dla mieszkańców,
    \item zwiększenie automatyzacji procesów poprzez implementację zaawansowanych mechanizmów ich definiowania i monitorowania. % TO DO: ja wspomniałem o BPMN w kontekście metod modelowania procesów oraz ich wdrażania z wykorzystaniem silników przetwarzających opisy wykonane w tym języku. Pana wtrącenie o BPMN usunąłem, bo nic ono nie wnosiło, nie miało też zbytniego sensu (było wyrwane z kontekstu).
\end{itemize}



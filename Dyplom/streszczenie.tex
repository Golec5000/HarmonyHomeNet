\pdfbookmark[0]{Streszczenie}{streszczenie.1}
%\phantomsection
%\addcontentsline{toc}{chapter}{Streszczenie}
%%% Poniższe zostało niewykorzystane (tj. zrezygnowano z utworzenia nienumerowanego rozdziału na abstrakt)
%%%\begingroup
%%%\setlength\beforechapskip{48pt} % z jakiegoś powodu była maleńka różnica w położeniu nagłówka rozdziału numerowanego i nienumerowanego
%%%\chapter*{\centering Abstrakt}
%%%\endgroup
%%%\label{sec:abstrakt}
%%%Lorem ipsum dolor sit amet eleifend et, congue arcu. Morbi tellus sit amet, massa. Vivamus est id risus. Sed sit amet, libero. Aenean ac ipsum. Mauris vel lectus. 
%%%
%%%Nam id nulla a adipiscing tortor, dictum ut, lobortis urna. Donec non dui. Cras tempus orci ipsum, molestie quis, lacinia varius nunc, rhoncus purus, consectetuer congue risus. 
%\mbox{}\vspace{2cm} % można przesunąć, w zależności od długości streszczenia
\begin{abstract}
Praca inżynierska dotyczy projektu aplikacji internetowej przeznaczonej dla członków i administracji spółdzielni mieszkaniowych. Aplikacja ta pozwalać ma na automatyzację komunikacji między spółdzielnią a mieszkańcami, zapewniając jednocześnie bezpieczeństwo danych i elastyczność w dostępie do informacji. Powinna również dawać możliwość przeglądania umów, rozliczeń, zgłoszeń oraz komunikatów, a także na wykonywanie płatności online, wszystko w intuicyjnym i przyjaznym dla użytkownika środowisku.

W pracy przedstawiono szczegółową analizę funkcjonalności systemów eBOK. Opisano przyjęte założenia projektowe oraz sposób implementacji, testowania i wdrożenie wynikowej aplikacji. Zaprezentowano też stos wykorzystanych technologii: Spring Boot, Hibernate, Next.js oraz bazę danych PostgreSQL. Powstała aplikacja może być uruchomiona w środowisku produkcyjnym, usprawniając codzienną obsługę mieszkańców i poprawiając efektywność działania spółdzielni.
\end{abstract}
\mykeywords

{
\selectlanguage{english}
\begin{abstract}
The engineering work focuses on designing a Web application for members and the administration of housing cooperatives. The application allows communication automation between administration staff and residents, while ensuring data security and flexibility in accessing information. It should also provide the ability to view contracts, billing, notifications, and communications and to make payments online, all in an intuitive and user-friendly environment.

The work presents a detailed analysis of the functionality of electronic customer service. It describes the design assumptions made and how the resulting application was implemented, tested and deployed. It also presents the stack of technologies used: Spring Boot, Hibernate, Next.js and PostgreSQL database. The resulting application can be implemented in a working environment, improving the daily service of residents, and improving the efficiency of the cooperative.
\end{abstract}
\mykeywords
}

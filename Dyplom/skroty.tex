\pdfbookmark[0]{Skróty}{skroty.1}% 
%%\phantomsection
%%\addcontentsline{toc}{chapter}{Skróty}
\chapter*{Skróty}
\label{sec:skroty}
\noindent\vspace{-\topsep-\partopsep-\parsep} % Jeśli zaczyna się od otoczenia description, to otoczenie to ląduje lekko niżej niż wylądowałby zwykły tekst, dlatego wstawiano przesunięcie w pionie
\begin{description}[labelwidth=*]
  \item [eBOK] (pl.\ \emph{Elektroniczne Biuro Obsługi Klienta})
  \item [IT] (ang.\ \emph{Information Technology})
  \item [WWW] (ang.\ \emph{World Wide Web})
  \item [API] (ang.\ \emph{Application Programming Interface})
  \item [HTTPS] (ang.\ \emph{Hypertext Transfer Protocol Secure})
	\item [HTTP] (ang.\ \emph{Hypertext Transfer Protocol})
  \item [TLS] (ang.\ \emph{Transport Layer Security})
  \item [JSON] (ang.\ \emph{JavaScript Object Notation})
  \item [SQL] (ang.\ \emph{Structured Query Language})
  \item [PostgreSQL] (ang.\ \emph{Postgre Structured Query Language})
  \item [XSS] (ang.\ \emph{Cross-Site Scripting})
\end{description}

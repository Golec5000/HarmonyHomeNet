\chapter{Wstęp}
\section{Wprowadzenie}
Rola automatyzacji w obsłudze klienta znacznie wzrosła w ostatnich latach, co bezpośrednio doprowadziło do rozpowszechnienia elektronicznych biur obsługi klienta (eCSO). 
% TO DO: co z rozwinięciem skrótu? Czy to nie " European Cyber Security Organisation (ECSO)"?
Takie systemy są szeroko stosowane w różnych sektorach gospodarki, takich jak energetyka, wodociągi, telekomunikacja i usługi multimedialne. Główną zaletą jest elastyczność systemu, która pozwala użytkownikom na korzystanie z niego w dowolnym miejscu i czasie. Takie podejście nie tylko znacząco poprawia jakość obsługi klienta, ale także zwiększa efektywność organizacji wdrażających takie rozwiązanie.

Elektroniczne biura obsługi klienta nie tylko umożliwiają użytkownikom zdalny dostęp do informacji o rozliczeniach i płatnościach, ale także wspierają bardziej zaawansowane procesy, takie jak zgłaszanie problemów technicznych, zarządzanie zgłoszeniami serwisowymi i komunikację z zarządem. Ważnym aspektem tych systemów jest bezpieczeństwo danych osobowych, które jest zapewniane poprzez regularne aktualizacje systemu, wdrażanie mechanizmów uwierzytelniania dwuskładnikowego oraz zgodność z przepisami o ochronie danych osobowych, takimi jak RODO.

%\subsection{Istniejące rozwiązania na rynku} % Tworzenie podrozdziału nie jest tutaj konieczne

Obecnie na rynku istnieje szereg rozwiązań eBOK, które są szeroko stosowane w różnych sektorach. Jednym z przykładów jest system „Energa24”, który umożliwia klientom dostęp do informacji o zużyciu energii, przeglądanie i pobieranie rachunków, płatności online i zgłaszanie awarii~\cite{energa}. 
% TO DO: można to tutaj nieco rozwinąć (by pokazać podstawowe funkcje - tutaj ich zestaw jest nieco inny niż w ebokach spółdzielni mieszkaniowych - bo serwis jest głównie do obsługi rozliczeń energii na podstawie odczytów liczników - z historią fakturowania itp. podczas gdy w spółdzielniach są media, pożytki, podatki, opłaty śmieciowe itd.)

Innym szeroko stosowanym rozwiązaniem jest eBOK dostarczany przez Miejskie Przedsiębiorstwo Wodociągów i Kanalizacji (MPWiK) w Warszawie, który umożliwia mieszkańcom monitorowanie zużycia wody, zarządzanie płatnościami i zgłaszanie problemów technicznych.
% TO DO: proszę dodać cytowania do wskazanych przykładów (tj. do manuali, podręczów użytkownika, stron domowych projektów)
% TO DO: można też dodać jakieś zrzuty z ekranu (na prawie cytatu)

Na rynku dostępne są również rozwiązania dostosowane specjalnie do potrzeb spółdzielni i wspólnot mieszkaniowych. Przykładem może być system „e-Kartoteka”, który umożliwia mieszkańcom zarządzanie zgłoszeniami usterek oraz podgląd postępu zgłoszenia, wgląd oraz regulację w płatności za nieruchomość, wgląd w dokumenty nieruchomości oraz wspólnoty, głosowanie nad uchwałami wspólnoty, komunikację z zarządem. 

Systemy eBOK różnią się funkcjonalnością, skalowalnością i stopniem integracji z innymi systemami, takimi jak narzędzia płatności online, systemy zarządzania nieruchomościami i aplikacje mobilne. W zależności od charakterystyki organizacji, wybór konkretnego rozwiązania będzie zależał od potrzeb operacyjnych, liczby użytkowników i stopnia automatyzacji procesów.
% TO DO: dodać zdanie/zdania łączące, coś podobnego do: wdrażanie takich systemów łączy się z koniecznością zastosowania różnych technologii. Projektowanie ich architektur stawia też liczne wyzwania. Jest w związku z tym dokonałym polem do sprawdzenia własnych umiejętności z dziedziny programowania. Dlatego postanowiono zająć się tym zagadnieniem w praktyce." 
% 
%\subsection{Motywacja i cele}
Tematem niniejszej pracy inżynierskiej jest projektowanie i wdrożenie aplikacji Elektronicznego Biura Obsługi Klienta (eBOK) dla wspólnoty mieszkaniowej. Wybór tego rozwiązania jest odpowiedzią na rosnącą potrzebę automatyzacji procesów obsługi klienta oraz usprawnienia komunikacji między mieszkańcami a administracją. Współczesne eBOK-i nie tylko eliminują konieczność kontaktu osobistego czy telefonicznego, ale także oferują użytkownikom możliwość załatwienia większości spraw online, co wpływa na wygodę, oszczędność czasu oraz bezpieczeństwo transakcji i komunikacji.


% Zwykle wprowadzenie zajmuje półtorej do dwóch stron

\section{Cel i zakres pracy}

% TO DO: dodać coś o zakresie pracy - od analizy przypadku, po projektowanie i wdrożenie.
% a więc trochę o architekturze (czy to będzie aplikacja webowa, dostępna przez przeglądarkę, czy może serwis, dostepny przez klientów dekstopowych i mobilnych, czy jeszcze coś innego), podstawowych technologiach, planowanym wdrożeniu 

% TO DO: podczas redakcji pracy pojawia się kłopot z czasami (przeszły, teraźniejszy, przyszły), czasownikami (dokonane, niedokonane) i trybami (orzekający, rozkazujący, przypuszczający oraz warunkowy). Zwykle podczas redagowania założeni stosuje się czasowniki niedokonane, tryp orzekający-warunkowy-przypuszczający. Proszę zwrócić na to uwagę.

Celem niniejszego opracowania jest zaprojektowanie i wdrożenie internetowej aplikacji elektronicznego biura obsługi klienta (eCSO) dla wspólnoty mieszkaniowej. Aplikacja ma ułatwić zarządzanie procesami związanymi z komunikacją, płatnościami, zgłoszeniami technicznymi i dostępem do odpowiednich dokumentów. W rezultacie zarówno mieszkańcy wspólnoty, jak i użytkownicy zarządzający skorzystają z kompleksowego narzędzia, które zautomatyzuje wiele codziennych zadań i poprawi efektywność ich praktyk.

Zakres prac objąć ma cały cykl rozwoju oprogramowania, począwszy od analizy potrzeb wspólnoty mieszkaniowej, poprzez zaprojektowanie architektury systemu, wdrożenie i przetestowanie gotowego rozwiązania. Aplikacja będzie dostępna w przeglądarce internetowej i zapewni dostęp z różnych urządzeń, w tym w wersji desktopowej. 

Backend systemu zostanie zrealizowany z wykorzystaniem technologii Java i Spring Boot, co umożliwi tworzenie skalowalnych i bezpiecznych aplikacji. Wykorzystaną bazą danych będzie PostgreSQL, uruchomiona w kontenerze dockerowym, podobnie jak i cały budowany system. Front-end aplikacji zostanie stworzony w języku TypeScript na platformie Next.js i zoptymalizowany pod kątem użyteczności i szybkości z nowoczesnym, responsywnym interfejsem użytkownika. 

\section{Układ pracy}
% tutaj opis zawartości kolejnych rozdziałów, można zredagować na końcu
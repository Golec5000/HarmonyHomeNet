\chapter{Wstęp}
\section{Wprowadzenie}
Tematem niniejszej pracy inżynierskiej jest projektowanie i wdrożenie aplikacji Elektronicznego Biura Obsługi Klienta (eBOK) dla wspólnoty mieszkaniowej. Wybór tego rozwiązania jest odpowiedzią na rosnącą potrzebę automatyzacji procesów obsługi klienta oraz usprawnienia komunikacji między mieszkańcami a administracją. Współczesne eBOK-i nie tylko eliminują konieczność kontaktu osobistego czy telefonicznego, ale także oferują użytkownikom możliwość załatwienia większości spraw online, co wpływa na wygodę, oszczędność czasu oraz bezpieczeństwo transakcji i komunikacji.

W pracy omówione zostaną podstawowe typy organizacji zarządzających nieruchomościami, takie jak wspólnoty mieszkaniowe i spółdzielnie, które różnią się strukturą prawną i organizacyjną. Zdecydowałem się na wybór wspólnoty mieszkaniowej jako szczególnego przypadku, który ma bardziej rozproszoną strukturę decyzyjną, a jednocześnie wymaga efektywnej komunikacji z mieszkańcami. Aplikacja będzie odpowiadać na potrzeby zarządzania wspólną przestrzenią, bieżących płatności oraz zgłoszeń technicznych. Wykorzystując nowoczesne rozwiązania webowe, system zostanie zoptymalizowany pod kątem użytkowników korzystających z Internetu.

W ostatnich latach automatyzacja obsługi klienta znacząco się rozwinęła, a eBOK-i zyskały szerokie zastosowanie w różnych branżach, takich jak energetyka, dostawy wody, czy usługi multimedialne. Systemy te oferują elastyczność działania – użytkownicy mogą korzystać z nich w dowolnym czasie, co znacząco wpływa na komfort obsługi. Dodatkowo, eBOK gwarantuje wysoki poziom bezpieczeństwa, m.in. dzięki regularnym aktualizacjom oraz implementacji takich funkcji jak uwierzytelnianie dwuskładnikowe, co jest kluczowe w kontekście ochrony danych osobowych.

Celem niniejszej pracy jest stworzenie aplikacji umożliwiającej zarówno mieszkańcom, jak i administratorom wspólnoty sprawne zarządzanie należnościami, komunikacją, zgłoszeniami, a także dostępem do dokumentów. System zostanie zaprojektowany tak, aby dostarczał kompleksowych informacji, automatyzował wiele procesów oraz poprawiał transparentność działań wspólnoty.

\section{Wymagania}

\subsection{Wymagania funkcjonalne}

\begin{enumerate}[label=\arabic*.]
    \item Zarządzanie kontem użytkownika
    \begin{itemize}
        \item Użytkownik jest w stanie się zarejestrować i zalogować.
        \item Użytkownik jest w stanie zarządzać danymi konta.
        \item Użytkownik jest w stanie odzyskać/zresetować hasło.
        \item Użytkownik jest w stanie usunąć konto.
        \item Administrator zarządza lokalami podpiętymi pod użytkownika.
    \end{itemize}
    \item Przegląd i płatność należności
    \begin{itemize}
        \item Użytkownik ma możliwość wglądu do bieżących i historycznych należności.
        \item Możliwość dokonywania płatności online za pomocą przelewu.
    \end{itemize}
    \item Obsługa zgłoszeń
    \begin{itemize}
        \item Użytkownik ma możliwość zgłaszania usterek oraz śledzenia ich statusu.
        \item Użytkownik ma możliwość komunikacji z administracją za pomocą systemu zgłoszeń.
    \end{itemize}
    \item Zarządzanie dokumentami
    \begin{itemize}
        \item Użytkownik ma możliwość przeglądu umów, regulaminów i innych dokumentów.
        \item Użytkownik ma możliwość pobierania faktur i umów w formacie PDF.
        \item Administracja ma możliwość archiwizacji dokumentów.
    \end{itemize}
    \item Zarządzanie głosowaniami
    \begin{itemize}
        \item Użytkownik ma możliwość udziału w głosowaniach swojej wspólnoty w zależności od uprawnień, czyli czy jest właścicielem lub najemcą.
        \item Użytkownik ma możliwość wglądu do wyników w czasie rzeczywistym.
        \item Administracja jest odpowiedzialna za tworzenie i nadzór przebiegu głosowania.
    \end{itemize}
    \item Powiadomienia
    \begin{itemize}
        \item System automatycznie, za zgodą użytkownika, będzie wysyłał powiadomienia e-mail oraz SMS o płatnościach, zgłoszeniach i innych ważnych sprawach.
    \end{itemize}
    \item Panel administracyjny
    \begin{itemize}
        \item Administracja jest odpowiedzialna za zarządzanie kontami użytkowników, monitoring zgłoszeń i raporty.
    \end{itemize}
    \item Kalkulator należności
    \begin{itemize}
        \item W systemie będzie proces odpowiedzialny za naliczanie należności dla każdego lokalu użytkownika w zależność od odpowiednich czynników.
    \end{itemize}
    \item Interakcje z zewnętrznymi systemami
    \begin{itemize}
        \item Pobieranie danych np. z liczników do obliczenia kosztów należności.
    \end{itemize}
\end{enumerate}

\subsection{Wymagania niefunkcjonalne}

\begin{enumerate}[label=\arabic*.]
    \item Skalowalność
    \begin{itemize}
        \item Aplikacja powinna obsługiwać zarówno małe, jak i duże ilości użytkowników.
    \end{itemize}
    \item Dostępność
    \begin{itemize}
        \item System dostępny 24/7, obsługiwany z dowolnego urządzenia z dostępem do Internetu.
    \end{itemize}
    \item Bezpieczeństwo
    \begin{itemize}
        \item Zastosowanie zaawansowanych algorytmów szyfrujących, zabezpieczenia przed atakami typu Injection, XXS.
        \item Dwuskładnikowe uwierzytelnianie (m.in. Google Authenticator).
    \end{itemize}
    \item Wydajność
    \begin{itemize}
        \item Krótki czas ładowania stron i danych.
    \end{itemize}
    \item Intuicyjność użytkowania interfejsu
    \begin{itemize}
        \item Prosty, nowoczesny design z widgetami ułatwiającymi dostęp do informacji o stanie konta, wykresów zużycia, dokumentów oraz innych pomocnych funkcji.
    \end{itemize}
    \item Integracja z zewnętrznymi usługami
    \begin{itemize}
        \item Integracja z operatorami płatności, urządzeniami pomagającymi obliczenie należności oraz usługami SMS/e-mail do wysyłania powiadomień.
    \end{itemize}
\end{enumerate}

\chapter{Wstęp}
\section{Wprowadzenie}
% TO DO: zacząć od opisu dziedziny
W ostatnich latach automatyzacja obsługi klienta znacząco się rozwinęła, a eBOK-i zyskały szerokie zastosowanie w różnych branżach, takich jak energetyka, dostawy wody, czy usługi multimedialne. Systemy te oferują elastyczność działania – użytkownicy mogą korzystać z nich w dowolnym czasie, co znacząco wpływa na komfort obsługi. Dodatkowo, eBOK gwarantuje wysoki poziom bezpieczeństwa, m.in.\ dzięki regularnym aktualizacjom oraz implementacji takich funkcji jak uwierzytelnianie dwuskładnikowe, co jest kluczowe w kontekście ochrony danych osobowych.

% TO DO: wskazać na istniejące rozwiązania

% TO DO: przejść do uzasadnienia - motywacji do podjęcia tematu
Tematem niniejszej pracy inżynierskiej jest projektowanie i wdrożenie aplikacji Elektronicznego Biura Obsługi Klienta (eBOK) dla wspólnoty mieszkaniowej. Wybór tego rozwiązania jest odpowiedzią na rosnącą potrzebę automatyzacji procesów obsługi klienta oraz usprawnienia komunikacji między mieszkańcami a administracją. Współczesne eBOK-i nie tylko eliminują konieczność kontaktu osobistego czy telefonicznego, ale także oferują użytkownikom możliwość załatwienia większości spraw online, co wpływa na wygodę, oszczędność czasu oraz bezpieczeństwo transakcji i komunikacji.

% Zwykle wprowadzenie zajmuje półtorej do dwóch stron

\section{Cel i zakres pracy}
Celem niniejszej pracy jest stworzenie aplikacji umożliwiającej zarówno mieszkańcom, jak i administratorom wspólnoty sprawne zarządzanie należnościami, komunikacją, zgłoszeniami, a także dostępem do dokumentów. System zostanie zaprojektowany tak, aby dostarczał kompleksowych informacji, automatyzował wiele procesów oraz poprawiał transparentność działań wspólnoty.

% TO DO: dodać coś o zakresie pracy - od analizy przypadku, po projektowanie i wdrożenie.
% a więc trochę o architekturze (czy to będzie aplikacja webowa, dostępna przez przeglądarkę, czy może serwis, dostepny przez klientów dekstopowych i mobilnych, czy jeszcze coś innego), podstawowych technologiach, planowanym wdrożeniu 
W pracy omówione zostaną podstawowe typy organizacji zarządzających nieruchomościami, takie jak wspólnoty mieszkaniowe i spółdzielnie, które różnią się strukturą prawną i organizacyjną. Zdecydowałem się na wybór wspólnoty mieszkaniowej jako szczególnego przypadku, który ma bardziej rozproszoną strukturę decyzyjną, a jednocześnie wymaga efektywnej komunikacji z mieszkańcami. Aplikacja będzie odpowiadać na potrzeby zarządzania wspólną przestrzenią, bieżących płatności oraz zgłoszeń technicznych. Wykorzystując nowoczesne rozwiązania webowe, system zostanie zoptymalizowany pod kątem użytkowników korzystających z Internetu.


\section{Układ pracy}
% tutaj opis zawartości kolejnych rozdziałów, można zredagować na końcu
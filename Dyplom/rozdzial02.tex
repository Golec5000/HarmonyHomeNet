\chapter{Analiza wymagań}
% TO DO: dodać jakieś zagajenie
\section{Zarys architektury systemu}
% TO DO: przedstawić (na diagramach poglądowych), jak ma wyglądać architektura systemu (z jakich komponentów będzie się ona składać, kto z niej będzie korzystać, jakie interfejsy będzie udostępniać). Może z tego będzie trzeba zrobić osobny rozdział.

\section{Wymagania funkcjonalne}
% TO DO: chyba na początek warto wyróżnić, w jakich rolach występować będą użytkownicy systemu i w ogóle - do czego ma służyć tworzona aplikacja

% Proszę zastanowić się, czy jednak nie dałoby się poniższego opisu jakoś rozwinąć (żeby nie był jedynie listą wyliczeniową). Czytelnik powinien zrozumieć, po co i dlaczego zdefiniowano poniższe wymagania. 

% Proszę zwrócić uwagę na "grupowanie" wymagań. Napisał Pan "Administrator zarządza lokalami ..." - ale trudno znaleźć wyjaśnienie, czym są te lokale i w ogóle - jaki to przypadek użycia (jaki jest kontekst, w jakim działać ma aplikacja).

\begin{enumerate}[label=\arabic*.]
    \item Zarządzanie kontem użytkownika
    \begin{itemize}
        \item Użytkownik jest w stanie się zarejestrować i zalogować.
        \item Użytkownik jest w stanie zarządzać danymi konta.
        \item Użytkownik jest w stanie odzyskać/zresetować hasło.
        \item Użytkownik jest w stanie usunąć konto.
        \item Administrator zarządza lokalami podpiętymi pod użytkownika.
    \end{itemize}
    \item Przegląd i płatność należności
    \begin{itemize}
        \item Użytkownik ma możliwość wglądu do bieżących i historycznych należności.
        \item Możliwość dokonywania płatności online za pomocą przelewu.
    \end{itemize}
    \item Obsługa zgłoszeń
    \begin{itemize}
        \item Użytkownik ma możliwość zgłaszania usterek oraz śledzenia ich statusu.
        \item Użytkownik ma możliwość komunikacji z administracją za pomocą systemu zgłoszeń.
    \end{itemize}
    \item Zarządzanie dokumentami
    \begin{itemize}
        \item Użytkownik ma możliwość przeglądu umów, regulaminów i innych dokumentów.
        \item Użytkownik ma możliwość pobierania faktur i umów w formacie PDF.
        \item Administracja ma możliwość archiwizacji dokumentów.
    \end{itemize}
    \item Zarządzanie głosowaniami
    \begin{itemize}
        \item Użytkownik ma możliwość udziału w głosowaniach swojej wspólnoty w zależności od uprawnień, czyli czy jest właścicielem lub najemcą.
        \item Użytkownik ma możliwość wglądu do wyników w czasie rzeczywistym.
        \item Administracja jest odpowiedzialna za tworzenie i nadzór przebiegu głosowania.
    \end{itemize}
    \item Powiadomienia
    \begin{itemize}
        \item System automatycznie, za zgodą użytkownika, będzie wysyłał powiadomienia e-mail oraz SMS o płatnościach, zgłoszeniach i innych ważnych sprawach.
    \end{itemize}
    \item Panel administracyjny
    \begin{itemize}
        \item Administracja jest odpowiedzialna za zarządzanie kontami użytkowników, monitoring zgłoszeń i raporty.
    \end{itemize}
    \item Kalkulator należności
    \begin{itemize}
        \item W systemie będzie proces odpowiedzialny za naliczanie należności dla każdego lokalu użytkownika w zależność od odpowiednich czynników.
    \end{itemize}
    \item Interakcje z zewnętrznymi systemami
    \begin{itemize}
        \item Pobieranie danych np. z liczników do obliczenia kosztów należności.
    \end{itemize}
\end{enumerate}

\section{Wymagania niefunkcjonalne}

\begin{enumerate}[label=\arabic*.]
    \item Skalowalność
    \begin{itemize}
        \item Aplikacja powinna obsługiwać zarówno małe, jak i duże ilości użytkowników.
    \end{itemize}
    \item Dostępność
    \begin{itemize}
        \item System dostępny 24/7, obsługiwany z dowolnego urządzenia z dostępem do Internetu.
    \end{itemize}
    \item Bezpieczeństwo
    \begin{itemize}
        \item Zastosowanie zaawansowanych algorytmów szyfrujących, zabezpieczenia przed atakami typu Injection, XXS.
        \item Dwuskładnikowe uwierzytelnianie (m.in. Google Authenticator).
    \end{itemize}
    \item Wydajność
    \begin{itemize}
        \item Krótki czas ładowania stron i danych.
    \end{itemize}
    \item Intuicyjność użytkowania interfejsu
    \begin{itemize}
        \item Prosty, nowoczesny design z widgetami ułatwiającymi dostęp do informacji o stanie konta, wykresów zużycia, dokumentów oraz innych pomocnych funkcji.
    \end{itemize}
    \item Integracja z zewnętrznymi usługami
    \begin{itemize}
        \item Integracja z operatorami płatności, urządzeniami pomagającymi obliczenie należności oraz usługami SMS/e-mail do wysyłania powiadomień.
    \end{itemize}
\end{enumerate}

\chapter{Analiza wymagań}
Analiza wymagań to kluczowy etap procesu tworzenia oprogramowania, który ma na celu zrozumienie potrzeb użytkowników oraz opracowanie poprawnej specyfikacji systemu. Oprócz wymagań funkcjonalnych, analiza powinna uwzględniać również aspekty związane ze skalowalnością, bezpieczeństwem i utrzymaniem systemu. W przypadku projektowania aplikacji typu eBOK dla wspólnoty mieszkaniowej istotne jest zidentyfikowanie wszystkich niezbędnych funkcji, które będą odpowiadały zarówno mieszkańcom, jak i administratorom. Ponadto, kluczowe zagadnienia, takie jak bezpieczeństwo, wydajność, intuicyjność interfejsu, architektura oraz integracja z innymi systemami, muszą zostać dokładnie przeanalizowane.

\section{Architektura systemu} 
Zaprojektowanie architektury systemu przed jego implementacją umożliwia stworzenie efektywnego i funkcjonalnego rozwiązania. Proces projektowania architektury zazwyczaj rozpoczyna się od szkicowania jej zarysu, z wyszczególnieniem głównych komponentów systemu oraz ich wzajemnych interakcji. Taki szkic przedstawiono na Rysunku~\ref{fig:zarys_architektury}, który pokazuje współpracujące ze sobą komponenty wdrożone przy użyciu technologii konteneryzacji.

\begin{figure}[ht]
    \centering
    \includegraphics[width=\linewidth]{rys02/zarys_architektury}
    \caption{Zarys architektury systemu „Harmony Home Net”}
    \label{fig:zarys_architektury}
\end{figure}

% DONE: Udało mi się chyba w odpowiedni sposób odnieść się do zaleceń z rodziału 5 
\subsection{Komponenty systemu}
System składa się z następujących elementów, przedstawionych na schemacie (rys. \ref{fig:zarys_architektury}):

\begin{enumerate}[label=\arabic*.]

	\item \textbf{Interfejs użytkownika (Frontend)} – Warstwa prezentacyjna aplikacji, zrealizowana przy użyciu frameworka Next.js (TypeScript), odpowiada za interakcje z użytkownikami oraz prezentację danych. Na schemacie (rys. \ref{fig:zarys_architektury}) warstwa ta została umieszczona po prawej stronie, gdzie obsługuje mieszkańców oraz administratorów. Komunikacja z backendem odbywa się przy użyciu bezpiecznego protokołu HTTPS, który szyfruje dane za pomocą protokołu TLS.

	\item \textbf{Serwer (Backend)} – Centralny moduł aplikacyjny, oparty na Spring Boot, uruchamiany w kontenerze Docker. Odpowiada on za zarządzanie logiką biznesową oraz przetwarzanie danych. Na schemacie (rys. \ref{fig:zarys_architektury}) serwer został umieszczony w centrum, pośrednicząc między interfejsem użytkownika, bazą danych i zewnętrznymi systemami. Warstwa ta realizuje zadania takie jak autoryzacja użytkowników (OAuth 2.0), zarządzanie zgłoszeniami technicznymi oraz obsługa transakcji finansowych. Konteneryzacja za pomocą Dockera zapewnia elastyczność oraz ułatwia skalowanie.

	\item \textbf{Logika biznesowa} – Odpowiada za realizację procesów biznesowych, koordynując przepływ danych pomiędzy warstwami systemu, takimi jak interfejs użytkownika, baza danych oraz zewnętrzne systemy. Na schemacie (rys. \ref{fig:zarys_architektury}) warstwa logiki biznesowej została wyraźnie wyodrębniona i pełni kluczową rolę w przetwarzaniu zgłoszeń technicznych, zarządzaniu danymi użytkowników oraz integracji z systemami zewnętrznymi.

	\item \textbf{Warstwa dostępu do danych} – Odpowiada za operacje na bazie danych PostgreSQL, która przechowuje kluczowe informacje dotyczące użytkowników, zgłoszeń oraz transakcji. Na schemacie (rys. \ref{fig:zarys_architektury}) system bazodanowy został umieszczony po prawej stronie i bezpośrednio połączony z modułem serwerowym, co umożliwia efektywne zarządzanie danymi.

	\item \textbf{System bazodanowy} – Aplikacja korzysta z obrazu PostgreSQL uruchamianego w kontenerze Docker. Na schemacie (rys. \ref{fig:zarys_architektury}) baza danych PostgreSQL została przedstawiona jako kluczowy element infrastruktury, odpowiedzialny za przechowywanie wszystkich danych aplikacji. Konteneryzacja bazy danych pozwala na łatwe zarządzanie środowiskiem oraz zwiększa bezpieczeństwo i skalowalność~\cite{Docker-docs,vsupalov}.

	\item \textbf{Moduł autoryzacji (OAuth 2.0)} – Umożliwia bezpieczny dostęp do zasobów systemu, zarządzając autoryzacją użytkowników. Na schemacie (rys. \ref{fig:zarys_architektury}) serwer OAuth 2.0 został pokazany jako kluczowy element komunikacji pomiędzy warstwą serwerową a użytkownikami, kontrolując uprawnienia i dostęp do danych.

	\item \textbf{Systemy zewnętrzne} – Aplikacja integruje się z zewnętrznymi systemami, takimi jak systemy bankowe czy urządzenia pomiarowe (np. liczniki zużycia mediów). Jak przedstawiono na Rysunku \ref{fig:zarys_architektury}, systemy te współpracują z backendem, wymieniając dane za pośrednictwem API.

	
\end{enumerate}

\subsection{Interakcje między komponentami}

Interakcje między komponentami są przedstawione na schemacie (\ref{fig:zarys_architektury}) i stanowią kluczowy element funkcjonowania aplikacji. Konteneryzacja Dockera umożliwia izolację poszczególnych modułów, co sprzyja ich łatwej skalowalności i zarządzaniu.

\begin{enumerate}[label=\arabic*.]

	\item \textbf{Interfejs użytkownika i Serwer} – Moduł prezentacyjny (interfejs użytkownika) komunikuje się z warstwą serwerową poprzez żądania HTTPS, szyfrowane protokołem TLS. Na schemacie (rys. \ref{fig:zarys_architektury}) połączenie to jest oznaczone ikoną „Encrypted data” oraz ikoną „Checklist security”, symbolizującą kontrolę ról i uprawnień użytkowników podczas przetwarzania danych. Warstwa serwerowa przetwarza zapytania i odsyła odpowiedzi w formacie JSON.

	\item \textbf{Moduł serwerowy i baza danych} – Warstwa serwerowa łączy się z bazą danych PostgreSQL uruchomioną w kontenerze Docker. Na schemacie (rys. \ref{fig:zarys_architektury}) interakcja ta została przedstawiona jako bezpośrednie połączenie między tymi dwoma komponentami, co umożliwia bezproblemowe wykonywanie operacji odczytu i zapisu danych.

	\item \textbf{Zarządzanie płatnościami} – Moduł serwerowy integruje się z systemami płatności, zapewniając obsługę transakcji finansowych. Schemat (rys. \ref{fig:zarys_architektury}) przedstawia ten proces po lewej stronie, gdzie systemy płatności wymieniają dane z warstwą serwerową, umożliwiając aktualizację statusu transakcji w bazie danych.

	\item \textbf{Powiadomienia dla użytkowników} – Powiadomienia są generowane bezpośrednio przez serwer i przesyłane do interfejsu użytkownika jako proste informacje, np. o zgłoszeniach, płatnościach lub głosowaniach. Na schemacie (rys. \ref{fig:zarys_architektury}) komunikacja ta została przedstawiona jako przepływ danych między serwerem a interfejsem użytkownika, co umożliwia informowanie użytkowników o istotnych wydarzeniach w aplikacji. Powiadomienia są wyświetlane w interfejsie użytkownika bez potrzeby korzystania z osobnego systemu powiadomień.

	
\end{enumerate}


\section{Wymagania aplikacji}

Aplikacja eBOK ma za zadanie wspierać zarządzanie wspólnotami mieszkaniowymi poprzez poprawę komunikacji między mieszkańcami a administracją oraz automatyzację procesów związanych z obsługą nieruchomości. Kluczowym elementem systemu jest zarządzanie lokalami, które odgrywają centralną rolę w wielu funkcjach aplikacji – od zgłaszania usterek, przez płatności, aż po organizację głosowań.

Lokale, definiowane jako nieruchomości (mieszkania, apartamenty, biura itp.), są głównym zasobem, który wymaga skutecznego zarządzania zarówno z perspektywy mieszkańca, jak i administratora. Dla mieszkańca lokal stanowi jednostkę przypisania płatności, zgłoszeń technicznych i dokumentów, natomiast dla administratora stanowi obszar monitorowania bieżących płatności, obsługi technicznej oraz interakcji z właścicielami i najemcami. Zarządzanie lokalami w aplikacji eBOK będzie kluczową funkcją, umożliwiającą efektywne administrowanie nieruchomościami.

\subsection{Przykłady użycia aplikacji}

Aplikacja eBOK oferuje szeroki zakres funkcji, z których korzystają różne grupy użytkowników. Główne przypadki użycia obejmują zarządzanie relacjami pomiędzy mieszkańcami a administracją, w zakresie takich działań jak płatności, zgłoszenia usterek, dostęp do dokumentów oraz udział w głosowaniach. System przewiduje dwie główne grupy użytkowników:

\begin{itemize} 
	\item \textbf{Mieszkańcy (właściciele i najemcy)}: Po zalogowaniu do systemu, użytkownicy mają możliwość przeglądania należności przypisanych do lokalu, regulowania opłat online oraz przeglądania historii rachunków. Mogą również zgłaszać usterki (np. awarie ogrzewania, przecieki) i monitorować status zgłoszeń. Dodatkowo, mają dostęp do dokumentów związanych z lokalem, takich jak umowy, regulaminy czy faktury.
	
	\item \textbf{Administratorzy}: Korzystają z panelu administracyjnego, który umożliwia zarządzanie zgłoszeniami technicznymi poprzez przypisywanie zadań wykonawcom oraz monitorowanie płatności związanych z każdym lokalem. Administratorzy prowadzą także rejestr dokumentów i utrzymują kontakt z mieszkańcami.

\end{itemize}

Każda z tych grup użytkowników korzysta z funkcji dostosowanych do ich potrzeb.

\subsection{Wymagania funkcjonalne}

Wymagania funkcjonalne aplikacji eBOK koncentrują się na kluczowych działaniach, które zapewniają sprawne funkcjonowanie wspólnot mieszkaniowych.

\begin{enumerate}[label=\arabic*.]

    \item \textbf{Zarządzanie kontem użytkownika i lokalami:} Użytkownicy muszą mieć możliwość zakładania kont, zarządzania swoimi danymi osobowymi oraz przypisywania lokali do konta. Mieszkańcy mogą być przypisani do wielu lokali, w zależności od tego, czy są właścicielami, czy najemcami. Funkcje dostępne w systemie zależą od roli użytkownika w danym lokalu (właściciel ma dostęp do szerszego zakresu funkcji niż najemca).

	\item \textbf{Przegląd i płatność należności:} Aplikacja umożliwia mieszkańcom przeglądanie szczegółowych informacji na temat należności związanych z danym lokalem. Właściciele mają dostęp do pełnej historii opłat oraz możliwość realizacji płatności online. Najemcy mogą przeglądać jedynie opłaty wynikające z umowy najmu. Powiadomienia o zaległościach przypominają użytkownikom o konieczności uregulowania opłat.

	\item \textbf{Zarządzanie zgłoszeniami technicznymi dla lokali:} Mieszkańcy mogą zgłaszać problemy techniczne dotyczące lokali, które są im przypisane. Właściciele mogą mieć dodatkowe uprawnienia do nadzorowania zgłoszeń związanych z większą liczbą lokali.

	\item \textbf{Zarządzanie dokumentami:} Lokale są powiązane z dokumentami, takimi jak umowy najmu, protokoły przekazania, regulaminy czy faktury. Właściciele mają pełen dostęp do dokumentów związanych z ich lokalami, natomiast najemcy mogą przeglądać tylko te dokumenty, które dotyczą ich umowy najmu. Administratorzy zarządzają dokumentami, przypisując je odpowiednim użytkownikom.

	\item \textbf{Głosowania mieszkańców:} W przypadku decyzji dotyczących zarządzania wspólnotą, administrator może organizować głosowania. Prawo do głosowania mają mieszkańcy przypisani do lokali, na podstawie swojej roli (właściciele). Proces głosowania jest zautomatyzowany, a wyniki są dostępne w czasie rzeczywistym.

	\item \textbf{Powiadomienia i komunikacja:} System powiadamia użytkowników o ważnych wydarzeniach związanych z ich lokalem, takich jak zaległe płatności, zgłoszenia serwisowe czy głosowania. Funkcje powiadomień zależą od roli użytkownika (właściciel lub najemca) i dostępnych dla niego opcji.

	\item \textbf{Panel administracyjny:} Administratorzy korzystają z zaawansowanego panelu, który umożliwia zarządzanie wszystkimi lokalami, przypisywanie mieszkańców do lokali, weryfikację uprawnień użytkowników oraz monitorowanie zgłoszeń technicznych. Na podstawie umów lub dokumentów, administrator przypisuje lokale do kont użytkowników.

	\item \textbf{Integracja z zewnętrznymi systemami:} Aplikacja eBOK musi integrować się z zewnętrznymi systemami, takimi jak systemy pomiarowe liczników mediów (np. wodomierze, liczniki energii) oraz systemy płatności. Integracja umożliwia automatyczne generowanie rachunków oraz synchronizację danych dotyczących zużycia mediów.

	\item \textbf{Stany liczników:} Użytkownicy mogą wprowadzać i przeglądać stany liczników powiązanych z lokalami, takich jak odczyty wodomierzy czy liczników energii. System automatycznie generuje rozliczenia na podstawie wprowadzonych odczytów, co pozwala na dokładne monitorowanie zużycia mediów i kontrolowanie kosztów.

	\item \textbf{Forum mieszkańców:} System umożliwia mieszkańcom uczestnictwo w forum dyskusyjnym. Każdy mieszkaniec przypisany do lokalu, niezależnie od roli (właściciel, najemca), może uczestniczyć w dyskusjach, dodawać nowe tematy oraz odpowiadać na pytania innych mieszkańców. Administratorzy moderują forum i mogą przypinać istotne informacje.

\end{enumerate}

\subsection{Wymagania niefunkcjonalne}

\begin{enumerate}[label=\arabic*.]

   \item \textbf{Skalowalność:} System musi być elastyczny, aby obsługiwać zarówno małe wspólnoty, jak i duże spółdzielnie, zarządzające setkami lub tysiącami lokali. Zastosowanie technologii Docker do konteneryzacji umożliwia skalowanie poziome (dodawanie instancji), co pozwala na efektywne rozdzielenie obciążenia.

	\item \textbf{Bezpieczeństwo:} Wszystkie dane muszą być przechowywane i przesyłane w sposób bezpieczny. Wymagana jest komunikacja szyfrowana (TLS) oraz autoryzacja za pomocą OAuth 2.0. System musi także chronić dane przed nieautoryzowanym dostępem i atakami, np. SQL Injection czy XSS.

	\item \textbf{Wydajność:} System powinien zapewniać szybki czas odpowiedzi, szczególnie podczas przetwarzania dużych ilości danych, takich jak zgłoszenia techniczne czy płatności. Warstwa serwerowa oparta na Spring Boot oraz baza danych PostgreSQL muszą być odpowiednio zoptymalizowane.

	\item \textbf{Dostępność:} Aplikacja powinna być dostępna przez 99,9\% czasu, z minimalnymi przerwami na konserwację. Monitorowanie serwerów oraz wdrożenie technologii Docker umożliwia szybkie przywracanie usług w przypadku awarii.

	\item \textbf{Intuicyjność interfejsu:} Aplikacja musi być łatwa w obsłudze zarówno dla mieszkańców, jak i administratorów. Interfejs oparty na komponentach Shadcn UI, zgodny z WCAG 2.0, zapewnia dostępność i łatwość nawigacji.

	\item \textbf{Zgodność z przepisami RODO:} System musi spełniać wymagania RODO dotyczące ochrony danych osobowych. Użytkownicy muszą mieć możliwość wglądu w swoje dane, ich modyfikacji oraz usunięcia. Proces przetwarzania danych musi być udokumentowany, a dostęp do danych ograniczony.

	\item \textbf{Niezawodność:} System musi zapewniać ciągłość działania oraz regularne tworzenie kopii zapasowych bazy danych. Plan odzyskiwania po awarii musi umożliwiać szybkie przywrócenie funkcjonalności systemu.

	\item \textbf{Testowalność:} Wszystkie funkcje aplikacji powinny być testowane automatycznie, co pozwala na szybką identyfikację błędów. Testy jednostkowe, integracyjne i systemowe powinny obejmować zarówno warstwę serwerową, jak i interfejs użytkownika.

	\item \textbf{Elastyczność:} Architektura systemu musi umożliwiać łatwe wprowadzanie nowych funkcji oraz modyfikacje istniejących, bez konieczności szerokiej przebudowy kodu. Architektura warstwowa (N-tier) zapewnia rozdzielenie logiki biznesowej, danych oraz interfejsu użytkownika.

\end{enumerate}

\section{Podsumowanie}

Analiza wymagań dla systemu ,,Harmony Home Net’’ stanowi kluczowy krok w kierunku zaprojektowania i wdrożenia aplikacji, która sprosta potrzebom mieszkańców oraz administratorów wspólnot mieszkaniowych. Omówione w tym rozdziale wymagania funkcjonalne i niefunkcjonalne wyznaczają solidną podstawę do dalszego projektowania i planowania architektury systemu, gwarantując wysoką wydajność, bezpieczeństwo oraz skalowalność.

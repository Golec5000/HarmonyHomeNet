\chapter{Instrukcja wdrożeniowa}

W celu uruchomienia prototypu systemu „Harmony Home Net” przygotowano środowisko bazujące na technologii Docker, które umożliwia szybkie uruchomienie wszystkich niezbędnych komponentów, takich jak backend, frontend oraz baza danych. Poniżej przedstawiono szczegółowe wymagania i kroki związane z wdrożeniem.

\section{Wymagania wstępne}
Na maszynie wdrożeniowej powinny być zainstalowane:
\begin{itemize}
    \item Docker (wersja 20.10 lub nowsza),
    \item Docker Compose (wersja 1.29 lub nowsza),
    \item Konto Gmail skonfigurowane do wysyłania powiadomień (wymagane w prototypie).
\end{itemize}

\section{Konfiguracja środowiska}
Konfigurację kontenerów systemu zamieszczono w pliku \texttt{docker-compose.yml}. Kluczowe ustawienia obejmują:
\begin{itemize}
    \item konfigurację bazy danych PostgreSQL,
    \item uruchomienie backendu z podanymi zmiennymi środowiskowymi,
    \item uruchomienie frontendu w trybie deweloperskim,
    \item konfigurację narzędzia do zarządzania bazą danych (pgAdmin).
\end{itemize}

\begin{lstlisting}[language=yaml, caption=Fragment pliku \texttt{docker-compose.yml}]
version: '3.8'

services:
  backend_app:
    build:
      context: ./backend
    ports:
      - "8444:8443"
    depends_on:
      - db
    environment:
      # Database
      SPRING_DATASOURCE_URL: jdbc:postgresql://db:5432/HarmonyHomeNet_DB
      SPRING_DATASOURCE_USERNAME: user
      SPRING_DATASOURCE_PASSWORD: admin

      # Mail
      SPRING_MAIL_HOST: smtp.gmail.com
      SPRING_MAIL_PORT: 587
      SPRING_MAIL_USERNAME: your_email@example.com
      SPRING_MAIL_PASSWORD: yout_password

      # Super admin user details
      SUPER_ADMIN_FIRST_NAME: Daniel
      SUPER_ADMIN_LAST_NAME: Ryszkowski
      SUPER_ADMIN_EMAIL: daniel.hhn.SA@gmail.com
      SUPER_ADMIN_PASSWORD: superadmin_password123
      SUPER_ADMIN_PHONE: 111111111

  db:
    image: postgres:alpine
    ports:
      - "5433:5432"
    environment:
      POSTGRES_USER: user
      POSTGRES_PASSWORD: admin
      POSTGRES_DB: HarmonyHomeNet_DB
\end{lstlisting}

\subsection{Backend}

Dockerfile backendu został zaprojektowany zgodnie z dobrymi praktykami, co pozwala na tworzenie wieloetapowych obrazów. Pierwszy etap buduje aplikację, a drugi uruchamia ją w zoptymalizowanym środowisku.

\begin{lstlisting}[language=docker, caption=Dockerfile backendu]
FROM maven:3.8.5-openjdk-17 AS build
WORKDIR /backend_app
COPY pom.xml .
RUN mvn dependency:go-offline
COPY src ./src
RUN mvn package -DskipTests

FROM openjdk:17-jdk-alpine
WORKDIR /app
COPY --from=build /backend_app/target/*.jar app.jar
EXPOSE 8443
ENTRYPOINT ["java", "-jar", "app.jar", "--server.port=8443"]
\end{lstlisting}

\subsection{Frontend}

Dockerfile frontendu działa w trybie deweloperskim, co oznacza, że środowisko jest zoptymalizowane pod kątem prac rozwojowych. Mimo że takie podejście działa w prototypie, w przyszłości należy przygotować zoptymalizowaną wersję produkcyjną.

\begin{lstlisting}[language=docker, caption=Dockerfile frontendu]
FROM node:20-alpine
WORKDIR /app
COPY package*.json ./
RUN npm install
COPY . .
EXPOSE 3000
CMD npm run dev
\end{lstlisting}


\section{Przed pierwszym uruchomieniem}
\subsection{Zmienne środowiskowe}
Przed pierwszym uruchomieniem systemu należy skonfigurować zmienne środowiskowe, które definiują dane dostępowe do serwera pocztowego oraz informacje o koncie superadministratora. Są one zdefiniowane w sekcji \texttt{environment} pliku \texttt{docker-compose.yml}. Poniżej przedstawiono ich szczegóły:

\begin{lstlisting}[language=yaml, caption=Konfiguracja zmiennych środowiskowych]
# Mail
SPRING_MAIL_HOST: smtp.gmail.com
SPRING_MAIL_PORT: 587
SPRING_MAIL_USERNAME: your_email@example.com
SPRING_MAIL_PASSWORD: your_password

# Super admin user details
SUPER_ADMIN_FIRST_NAME: Daniel
SUPER_ADMIN_LAST_NAME: Ryszkowski
SUPER_ADMIN_EMAIL: daniel.hhn.SA@gmail.com
SUPER_ADMIN_PASSWORD: superadmin_password123
SUPER_ADMIN_PHONE: 111111111
\end{lstlisting}

\textbf{Opis pól do konfiguracji:}
\begin{itemize}
    \item \texttt{SPRING\_MAIL\_HOST}, \texttt{SPRING\_MAIL\_PORT} -- domyślne wartości dla serwera SMTP usługi Gmail.
    \item \texttt{SPRING\_MAIL\_USERNAME}, \texttt{SPRING\_MAIL\_PASSWORD} -- dane logowania do konta Gmail, które będzie wykorzystywane do wysyłania wiadomości e-mail. Wartości te należy uzupełnić własnym adresem e-mail i hasłem (lub hasłem aplikacji).
    \item \texttt{SUPER\_ADMIN\_FIRST\_NAME}, \texttt{SUPER\_ADMIN\_LAST\_NAME}, \texttt{SUPER\_ADMIN\_EMAIL}, \texttt{SUPER\_ADMIN\_PASSWORD}, \texttt{SUPER\_ADMIN\_PHONE} -- dane pierwszego użytkownika z rolą superadministratora. Wartości te można dostosować według potrzeb organizacji.
\end{itemize}

\subsection{Konfiguracja konta Gmail do wysyłania e-maili}
Aby system mógł korzystać z serwera Gmail do wysyłania wiadomości e-mail, należy skonfigurować konto Gmail zgodnie z poniższymi krokami:

\begin{enumerate}
    \item Zaloguj się na swoje konto Gmail.
    \item Przejdź do ustawień konta, a następnie wybierz sekcję \texttt{Bezpieczeństwo}.
    \item Włącz opcję \texttt{Dostęp mniej bezpiecznych aplikacji} (jeśli jest dostępna).
    \item Alternatywnie, jeżeli dostęp mniej bezpiecznych aplikacji nie jest dostępny, wygeneruj hasło aplikacji:
    \begin{itemize}
        \item Przejdź do sekcji \texttt{Hasła aplikacji}, przez wyszukiwarkę w panelu konta.
        \item Wybierz aplikację \texttt{Poczta} i urządzenie \texttt{Komputer}.
        \item Kliknij \texttt{Generuj} i skopiuj wygenerowane hasło.
        \item Wprowadź to hasło w polu \texttt{SPRING\_MAIL\_PASSWORD} w pliku \texttt{docker-compose.yml}.
    \end{itemize}
    \item \textbf{Upewnij się, że konto Gmail ma włączone uwierzytelnianie dwuskładnikowe, jeśli wymaga tego konfiguracja hasła aplikacji}.
\end{enumerate}

\section{Pierwsze uruchomienie systemu}
Po skonfigurowaniu zmiennych środowiskowych i konta Gmail, system można uruchomić za pomocą następującego polecenia:
\begin{verbatim}
docker-compose up --build
\end{verbatim}

W trakcie pierwszego uruchomienia backend automatycznie utworzy konto superadministratora z wykorzystaniem danych wprowadzonych w sekcji \texttt{SUPER\_ADMIN\_*} pliku \texttt{docker-compose.yml}. Konto to umożliwia dostęp do panelu administracyjnego aplikacji i zarządzanie systemem. Kod odpowiedzialny za inicjalizację prezentuje się następująco:
\begin{lstlisting}[language=Java, style=JavaStyle, caption=Kod inicjalizacji superadministratora]
@PostConstruct
public void init() {
    User superAdmin = createUserIfNotExists(
        superAdminFirstName,
        superAdminLastName,
        superAdminEmail,
        bCryptPasswordEncoder.encode(superAdminPassword),
        superAdminPhone,
        Role.ROLE_SUPER_ADMIN
    );
}
\end{lstlisting}

Po uruchomieniu:
\begin{itemize}
    \item Backend dostępny jest pod adresem \texttt{https://localhost:8444}.
    \item Frontend można otworzyć pod adresem \texttt{http://localhost:3000}.
    \item pgAdmin dostępny jest pod adresem \texttt{http://localhost:5050}.
\end{itemize}

\section{Uwagi i rozwój przyszłościowy}
Obecna implementacja frontendu w trybie deweloperskim działa poprawnie, ale wymaga optymalizacji przed wdrożeniem produkcyjnym. W przyszłości należy:
\begin{itemize}
    \item przygotować zoptymalizowaną wersję frontendu w trybie produkcyjnym,
    \item wdrożyć rzeczywistą integrację z zewnętrznymi systemami bankowymi i usługami SMS,
    \item rozszerzyć dokumentację wdrożeniową o instrukcje dotyczące monitorowania i skalowania aplikacji.
\end{itemize}
